\documentclass[]{article}
\usepackage[margin=1in]{geometry}
\usepackage{amsmath}
\usepackage{amssymb}
\usepackage{amsthm}

% Define the main theorem style
\newtheorem{lemmax}{Lemma}[section]

% Define the custom counter for sub-lemmas
\newcounter{sublemma}[lemmax]
\renewcommand{\thesublemma}{\thesection.1.\alph{sublemma}}

% Define a new environment for sub-lemmas
\newenvironment{sublemma}{
    \refstepcounter{sublemma} % Step the sub-lemma counter
    \noindent\textbf{Lemma \thesublemma.} % Print the sub-lemma number
}{}

\title{\vspace{-5em}{Chapter 6}}
\author{Sterling Jeppson}
\date{\today}

\begin{document}
\maketitle

% Start a new section numbered as 6
\setcounter{section}{6}

\begin{sublemma}
$\forall x \in \mathbb{R}$, $x \neq 3 \implies x^2 - 2x + 3 \neq 0$.
\begin{proof}
Let $x$ be any real number such that $x \neq 3$. We must show that $x^2 - 2x + 3 \neq 0$. Assume that it is equal to $0$ and 
derive a contradiction. Completing the square of $x^2 - 2x + 3 = 0$ we obtain
\begin{align*}
  x^2 - 2x + 3 &= 0\\
  (x^2 - 2x + 3) + 1& = 0 + 1\\
  (x^2 - 2x + 1) + 3 &=1 \\
  x^2 - 2x + 1 &= -2\\
  (x - 1)^2 &= -2
\end{align*}
which is a contradiction since $\forall y \in \mathbb{R}$,  $y^2 \geq 0$.
\end{proof}
\end{sublemma}

\begin{sublemma}
$\exists x \in \mathbb{C}$, $x \neq 3 \, \land \, x^2 - 2x + 3 = 0$.
\begin{proof}
When dealing with complex numbers the prior proposition does not hold. To show this we show that the negation is true.
Let $x = 1 + i\sqrt{2}$. This is not equal to 3 because the definition of equality on complex numbers is equal real and imaginary parts and $1 \neq 3$ and $\sqrt{2} \neq 0$.
Next we verify that $x^2 - 2x + 3 = 0$. 
\begin{align*}
  x^2 - 2x + 3 &= 0 \\
  (1 + i \sqrt{2})^2 - 2 (1 + i \sqrt{2}) + 3 &= 0 \\
  (1 + i \sqrt{2}) \cdot (1 + i \sqrt{2}) -2 - i \cdot 2 \sqrt{2} +3 &= 0 \\
  1 + i \cdot 2 \sqrt{2} + (i \sqrt{2})^2 - i \cdot 2 \sqrt{2} + 1 &= 0 \\
  (1 + 1) + (i \cdot 2 \sqrt{2} - i \cdot 2 \sqrt{2}) + (-1 \cdot 2) &= 0\\
  2 + 0 - 2 &= 0\\
  0 &= 0 \qedhere
\end{align*}
\end{proof}
\end{sublemma}

\begin{lemmax}
This is Lemma 6.2.
\begin{proof}
% Proof goes here
\end{proof}
\end{lemmax}

\begin{lemmax}
This is Lemma 6.3.
\begin{proof}
% Proof goes here
\end{proof}
\end{lemmax}

\end{document}
