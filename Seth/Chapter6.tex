\documentclass[]{article}
\usepackage[margin=1in]{geometry}
\usepackage{amsmath}
\usepackage{amssymb}
\usepackage{amsthm}

% Define the main theorem style
\newtheorem{lemmax}{Lemma}[section]

% Define the custom counter for sub-lemmas
\newcounter{sublemma}[lemmax]
\renewcommand{\thesublemma}{\thesection.1.\alph{sublemma}}

% Define a new environment for sub-lemmas
\newenvironment{sublemma}{
    \refstepcounter{sublemma} % Step the sub-lemma counter
    \noindent\textbf{Lemma \thesublemma.} % Print the sub-lemma number
}{}

\title{\vspace{-5em}{Chapter 6}}
\author{Sterling Jeppson}
\date{\today}

\begin{document}
\maketitle

% Start a new section numbered as 6
\setcounter{section}{6}

\setcounter{lemmax}{1}
\begin{sublemma}
$\forall x \in \mathbb{R}$, $x \neq 3 \implies x^2 - 2x + 3 \neq 0$.
\begin{proof}
Let $x$ be any real number such that $x \neq 3$. We must show that $x^2 - 2x + 3 \neq 0$. Assume that it is equal to $0$ and 
derive a contradiction. Completing the square of $x^2 - 2x + 3 = 0$ we obtain
\begin{align*}
  x^2 - 2x + 3 &= 0\\
  (x^2 - 2x + 3) + 1& = 0 + 1\\
  (x^2 - 2x + 1) + 3 &=1 \\
  x^2 - 2x + 1 &= -2\\
  (x - 1)^2 &= -2
\end{align*}
which is a contradiction since $\forall y \in \mathbb{R}$,  $y^2 \geq 0$.
\end{proof}
\end{sublemma}

\begin{sublemma}
$\exists x \in \mathbb{C}$, $x \neq 3 \, \land \, x^2 - 2x + 3 = 0$.
\begin{proof}
When dealing with complex numbers the prior proposition does not hold. To show this we show that the negation is true.
Let $x = 1 + i\sqrt{2}$. This is not equal to 3 because the definition of equality on complex numbers is equal real and imaginary parts and $1 \neq 3$ and $\sqrt{2} \neq 0$.
Next we verify that $x^2 - 2x + 3 = 0$. 
\begin{align*}
  x^2 - 2x + 3 &= 0 \\
  (1 + i \sqrt{2})^2 - 2 (1 + i \sqrt{2}) + 3 &= 0 \\
  (1 + i \sqrt{2}) \cdot (1 + i \sqrt{2}) -2 - i \cdot 2 \sqrt{2} +3 &= 0 \\
  1 + i \cdot 2 \sqrt{2} + (i \sqrt{2})^2 - i \cdot 2 \sqrt{2} + 1 &= 0 \\
  (1 + 1) + (i \cdot 2 \sqrt{2} - i \cdot 2 \sqrt{2}) + (-1 \cdot 2) &= 0\\
  2 + 0 - 2 &= 0\\
  0 &= 0 \qedhere
\end{align*}
\end{proof}
\end{sublemma}

\begin{lemmax}
$\forall n \in \mathbb{N}$, $2 < n < 3 \implies 7n + 3$ is odd.
\begin{proof}
The premise is false since there are not natural numbers between 2 and 3 and $n$ is a natural number. 
Therefore the conclusion follows vacuously. 
\end{proof}
\end{lemmax}

\begin{lemmax}
$\forall x \in \mathbb{Z}$, if $x$ is odd then $x^2$ is odd.
\begin{proof}
Let $x$ be some odd integer. Then $x = 2k + 1$ for some integer $k$. It follows that
\[ x^2 = (2k + 1)^2 = (4k^2 + 4k + 1) = 2 \cdot (2k^2 + 2k) + 1 \qedhere \]
\end{proof}
\end{lemmax}

\begin{lemmax}
$\forall x \in \mathbb{Z}$, if $x$ is even, then $7x - 5$ is odd.
\begin{proof}
Let $x$ be an even integer, so $x = 2k$ for some integer $k$. It follows that
\[
7x - 5 = 7(2k) - 5 = 14k - 5 = 2 \cdot (7k - 3) + 1 \qedhere
\]
\end{proof}
\end{lemmax}

\begin{lemmax}
$\forall a, b, c \in \mathbb{Z}$, if $a$ and $c$ are odd, then $a \cdot b + b \cdot c$ is even.
\begin{proof}
Since all integers are either even or odd $b$ must be either even or odd. If $b$ is even, then both $a \cdot b$ and $b \cdot c$ are even because the product of an even number with any integer is even. The sum of two even numbers is even, so $a \cdot b + b \cdot c$ is even.
If $b$ is odd, then both $a \cdot b$ and $b \cdot c$ are odd because the product of two odd numbers is odd. The sum of two odd numbers is even, so $a \cdot b + b \cdot c$ is even.
Thus, in both cases, $a \cdot b + b \cdot c$ is even.
\end{proof}
\end{lemmax}

\begin{lemmax}
$\forall n \in \mathbb{Z}$, $|n| < 1 \implies 3n - 2$ is even.
\begin{proof}
Given $|n| < 1$, the only integer that satisfies this inequality is $n = 0$. Substituting $n = 0$ into the expression, we have
\[
3 \cdot 0 - 2 = -2.
\]
Since $-2$ can be written as $2 \cdot (-1)$, it is an even number.
\end{proof}
\end{lemmax}

\begin{lemmax}
$\forall z \in \mathbb{Z}$, $z$ is odd $ \implies \exists \, a, c \in \mathbb{Z}, z = a^2 - c^2$.
\begin{proof}
Since $z$ is odd, we can write $z = 2k + 1$ for some integer $k$. Let $a = k + 1$ and $c = k$. Then,
\[
a^2 - c^2 = (k + 1)^2 - k^2.
\]
Expanding this, we get
\[
(k + 1)^2 - k^2 = (k^2 + 2k + 1) - k^2 = 2k + 1,
\]
which is exactly $z$. Therefore, there exist integers $a$ and $c$ such that $z = a^2 - c^2$.
\end{proof}
\end{lemmax}



\end{document}
